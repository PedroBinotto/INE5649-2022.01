\documentclass[11pt]{article}
\usepackage[portuguese]{babel}
\usepackage[T1]{fontenc}
\usepackage[a4paper, margin=2cm]{geometry}
\usepackage{authblk}
\usepackage{color}
\usepackage{listings}
\usepackage{setspace}
\usepackage{fontspec}
\usepackage{soul}
\usepackage{verbatimbox}

\setcounter{secnumdepth}{0}
\onehalfspacing
\setmainfont{Nimbus Sans}

\title{TRABALHO 1 - SEMESTRE 2022.2}
\author[1]{Pedro Santi Binotto [20200634]\thanks{\texttt{pedro.binotto@grad.ufsc.br}}}
\author[1]{Tales Antunes Mendes [20200636]\thanks{\texttt{talesmendes@hotmail.br}}}
\author[1]{Mateus Silva Teixeira [20200634]\thanks{\texttt{mateus.silva.teixeira@grad.ufsc.br}}}
\date{\today}

\affil[1]{Departamento de Informática e Estatística, Universidade Federal de Santa Catarina}

\begin{document}
\maketitle

\begin{abstract}
Este artigo tem o fim de estudar a relação entre o volume de iterações de
processamento realizados por uma aplicação e o tempo de computação necessário
para a conclusão da mesma, e à partir da análise dos dados resultantes,
comparar a performance da mesma tarefa implementada em diferentes linguagens de
programação, assim como projetar o tempo de processamento dos programas com um
número de iterações maior do que o volume utilizado nos testes de referência
através de modelos de regressão linear simples.
\end{abstract}

\newpage
\section{Introdução e Objetivos}
\paragraph{}
Historicamente, um dos maiores, se não o maior objetivo da pesquisa científica
na área da computação tem sido a otimização de performance de processamento,
tanto em termos de hardware\cite{schaller1997moore}, quanto na implementação de
software e linguagens de programação.

\paragraph{}
Em anos recentes, no entanto, avanços tecnológicos na construção de circuitos
de silício têm propiciado a utilização de linguagens de alto nível
\cite{srinath2017python}, que promovem a acessibilidade destas ferramentas
àqueles não familiarizados aos detalhes técnicos do desenvolvimento de software.

\paragraph{}
Todavia, a adoção destas facilidades representa, muitas vezes, um impacto
negativo na performance de computação de aplicações implementadas utilizando
estas linguagens\cite{prechelt2000empirical}.

\newpage
\section{Materiais e Métodos}
\paragraph{}

\newpage
\section{Resultados e Discussão}
\paragraph{}

\newpage
\section{Conclusões Finais}
\paragraph{}

\newpage
\section{Referências Bibliográficas}
\bibliographystyle{plain}
\bibliography{references}

\end{document}
